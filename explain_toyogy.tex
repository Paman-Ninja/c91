\newcommand*{\reflisting}[1]{\lstlistingname\ref{#1}}
\renewcommand*\descriptionlabel[1]{\normalfont\headfont #1 :\hfil}

\chapter{詳解:とよぎぃ通信}

とよぎぃ通信はタイプセットに{\pLaTeX}を用いています.
しかしながら,とよぎぃ通信のタイプセットに用いられているパッケージファイルは
様々な場所から持ってきた便利っぽい設定の集合体になっているために,
このままでは秘伝のタレになってしまいます.
先代の編集担当が厳しい編集スケジュールの中でパッチワーク的に生み出してしまったので仕方のないところもありますが.
というわけで,本稿ではパッケージファイルを秘伝のタレにしてしまわないために,タイプセットに用いるパッケージファイルを読み解き,可能な限りの解説を試みます.

参考にするとよぎぃ通信のタイプセット設定は第3号で用いたもの\footnote{\url{https://github.com/tut-cc/c90}}です.

\section{パッケージファイルを読み解く前に}
\subsection{簡単な{\TeX}マクロの説明}
\newcommand{\lcommand}{/コマンド/}
\newcommand{\lCommand}[1]{/#1/}
{
\def\command{/コマンド/}
\def\Command#1{/#1/}

パッケージファイルを読み解くにはまず{\TeX}および{\LaTeX}におけるマクロについての知識が必要です.
{\TeX}におけるマクロは次のように構文で定義します\footnote{正確に言えば{\TeX}のcontrol sequence}.
\begin{verbatim}
\def\command{/コマンド/}
\end{verbatim}
これで\verb|\command|というマクロが定義されました.
このように{\TeX}におけるマクロは\verb|\def\マクロ名{\置換文字列}|の形で定義されます.
次に,定義されたマクロを呼び出すことにします.
マクロを呼び出すには\verb|\マクロ名|の形で呼び出すマクロを指定します.
今回定義した\verb|command|マクロを呼び出してみます.
\begin{verbatim}
\command{}
\end{verbatim}
実際に呼び出してみると\texttt{\command{}}という出力が得られます.
これが{\TeX}の基本的なマクロの定義です.
このように{\TeX}のマクロは与えられたマクロ名を定義されている置換文字列へと置き換えるものであることが分かります.

次に,引数を取る{\TeX}のマクロ\verb|\Command|を定義します.
引数を取ることでマクロが置換文字列の内容を自由に変更することができます.
\begin{verbatim}
\def\Command#1{/#1/}
\end{verbatim}
先ほどのマクロと異なり,マクロ名の後ろに\verb|#1|という記述が追加されています.
これは,マクロが取る引数の数を示しており,マクロの置換文字列内部では\verb|#1|という名前で与えられた引数を参照することができます.
先ほどと同じように定義されたマクロを呼び出します.
\begin{verbatim}
\Command{コマンド呼び出し}
\end{verbatim}
実際に呼び出してみると\texttt{\Command{コマンド呼び出し}}という出力が得られます.
これで,引数を取る{\TeX}マクロを定義することができました.
引数が複数存在するときはそれぞれの引数を\{\}で囲むことで引数の範囲を明示します.

{\LaTeX}では{\TeX}とは異なる形でマクロを定義することができます\footnote{これも正確に言えば{\LaTeX}のUser-defined command}.
\verb|\command|と同様のマクロを{\LaTeX}における記法を用いて定義してみます.
\begin{verbatim}
\newcommand{\lcommand}{/コマンド/}
\end{verbatim}
}
呼び出し方は{\TeX}のマクロと何ら変わりません.呼び出してみると\texttt{\lcommand{}}という出力が得られます.
同じように\verb|Command|と同様のマクロを定義してみます.
\begin{verbatim}
\newcommand{\lCommand}[1]{/#1/}
\end{verbatim}
そして\verb|\lCommand{コマンド呼び出し}|とすると得られる出力は\texttt{\lCommand{コマンド呼び出し}}となります.
これが{\TeX}および{\LaTeX}のマクロの基本です.

\subsection{クラスそしてプリアンブル}

とよぎぃ通信は日本語文書なのでタイプセットには{\LaTeX}の日本語対応拡張である{\pLaTeX}を用いています.
{\pLaTeX}でのタイプセットに欠かせないものといえば,クラスファイル(\texttt{*.cls})とパッケージファイル(\texttt{*.sty})です.
一般に,クラスファイルで文書とレイアウトとの対応について定義をして,足りない部分をパッケージファイルで補うという形がとられます.
パッケージファイルはスタイルファイルとも呼ばれることがありますが,ここではややこしいのでパッケージファイルで統一します.

{\pLaTeX}で文書をタイプセットするときには,文書の先頭にクラスファイルを指定します.
とよぎぃ通信は同人誌ですから,使うクラスファイルは\texttt{jsclasses}に含まれる書籍用クラスファイル\texttt{jsbook}ということになります.
クラスファイルを読み込むときにはオプションを設定することができます.
このオプションによって,レイアウトをクラスファイルに沿った形でいくつか変更することができます.
第3号を発行する際には次のようなオプションを指定していました.
%\begin{lstlisting}[caption=クラスファイルの指定,label=lis:class,language=tex]
%\documentclass[papersize, b5paper, tombo, 11pt]{jsbook}
%\end{lstlisting}
%クラスファイル読み込み時に用いられているオプションの意味はそれぞれ次のような意味を持っています.
%papersize指定してるのにdvipdfmxで-p a4する理由したら意味がない気がするので調査
%description環境のアキがたいへんイケてないので後で直す
\begin{description}
	\item[papersize] DVI(DeVice Independent)ファイルの先頭にpapersizeスペシャル命令を書き込む
	\item[b5paper] 用紙サイズをB5に指定
	\item[tombo] 印刷会社に入稿するときに必要となるトンボの出力
	\item[11pt] 欧文フォントのサイズを11ptに指定
\end{description}

クラスファイルの読み込みから文書の開始までの領域がプリアンブルと呼ばれます.
ここに読み込むパッケージファイルを列挙していくことで機能の拡張を行うことができます.
\reflisting{lis:preamble}が第3号を刊行した時に用いたプリアンブルの指定です.
\begin{lstlisting}[caption=プリアンブル,label=lis:preamble,language=tex]
\usepackage[dvipdfmx]{graphicx}
\usepackage{comment}
\usepackage{url}
\usepackage{subfigure}
\usepackage{setspace}
\usepackage{amsmath}
\usepackage{multirow}
\usepackage{listings}
\usepackage{toyogy}
\end{lstlisting}
{\pLaTeX}でタイプセットを行ったことがある人ならば説明不要なものでしょう.
% それぞれ次のような意味を持っています.
% \begin{description}
% 	\item[graphicx] :
% 	\item[comment] :
% 	\item[url] :
% 	\item[subfigure] :
% 	\item[setspace] :
% 	\item[amsmath] :
% 	\item[multirow] :
% 	\item[listings] :
% 	\item[toyogy] とよぎぃ通信専用の設定が書き込まれたパッケージファイル
% \end{description}

これらの設定でとよぎぃ通信のタイプセットは行われています.というわけでこれで解説終了と言いたいところですが,ひとつだけ中身が全くわからないパッケージファイルがあります.
\texttt{toyogy.sty}こそがとよぎぃ通信を刊行する上での一番大事なパッケージファイルになのに,このままでは肝心の中身がさっぱりわかりません.
そこで,次節では\texttt{toyogy.sty}の中身について具体的な解説を施します.
\section{toyogy.sty}

とよぎぃ通信における設定の核を為すのが\texttt{toyogy.sty}です.
このパッケージファイルにはとよぎぃ通信を刊行するときに必要となる様々な設定が書き込まれています.
しかし,このパッケージファイルはパッチワークで作成されているために{\TeX}の機能の奥まで潜ったものから,ある既存パッケージに対しての設定を記述したものまで千差万別な内容が現れます.
このパッケージファイルを設定を作用させる対象ごとに切り分けて読んでいきます.

\subsection{ページレイアウトの設定}

本を作成する上で基本となるのが,用紙のどの部分にどの要素を割り当てて使うかということです.
ある本のページを構成する要素として次のようなものがあります.
\begin{itemize}
	\item 本文
	\item ページ番号
	\item 脚注
	\item 柱
\end{itemize}
これらの要素は過不足なく本の1ページ内に収まらなければなりません.
もし仮に設定を間違えてページ番号がページ内から抜け落ちてしまえば,読者は自分が何ページ目を読んでいるか把握することができませんし,編集する側もどのページを編集したのかわからなくなってしまいます.
それを防ぐために,クラスファイルがきちんとページ内に必要な要素が収まるように予めページレイアウトを設計しています\footnote{余談ですがページレイアウトのことは版面と呼んだりもします}.
しかし,その設定を少しだけ触ることでより理想に近づけたいということがあります.
とよぎぃ通信の場合もご多分に漏れず,クラスファイルの標準の設定からほんの少しだけ
ページレイアウトを修正しています.
%\reflisting{list:layout}が実際に用いたページレイアウトの修正です.
\begin{lstlisting}[caption = ページレイアウトの修正,label = list:layout,language = tex]
\setlength{\textwidth}{155truemm}
\setlength{\textheight}{230truemm}
\setlength{\fullwidth}{\textwidth}
\setlength{\oddsidemargin}{15truemm}
\addtolength{\oddsidemargin}{-1.15truein}
\setlength{\evensidemargin}{15truemm}
\addtolength{\evensidemargin}{-1.15truein}
\setlength{\topmargin}{30truemm}
\addtolength{\topmargin}{-2.4truein}
\addtolength{\footskip}{8truemm}
\pagestyle{plain}
\end{lstlisting}
何が何やらという感じなので,一つずつ見ていきましょう.

ページレイアウトの修正を理解するには,はじめに{\pLaTeX}がどのようにページレイアウトを行っているのかを知らねばなりません.
{\pLaTeX}では,用紙に対して文字を挿入するための領域が四つあり,それぞれ``ヘッダー'',``フッター'',``本文'',``傍注''と言います
\footnote{英語の場合はそれぞれ``header'', ``footer'', ``body'', ``margin note''}.
この中で最も重要なのが``本文''で,ここに文章が挿入されていくことになります.
``ヘッダー''は日本語書籍の場合, ``柱''とも呼ばれます.
ここに章番号や章タイトルを挿入して,現在のページがどの章に属しているかを明示します.
``フッター''にはノンブル,つまりページ番号が主に挿入されます.
``傍注''は普段あまり見ることはないかもしれませんが,文章の特定箇所に追加で説明を加えたい場合に``本文''領域の欄外に文章を挿入するために使われます.
それぞれの領域の配置は``本文''を中心として,``本文''上部に``ヘッダー'',下部に``フッター'',ページの小口側に``傍注''が配置されることになります.

\verb|\setlength|マクロを用いることでこれらの領域の大きさを変更します.
\verb|\setlength|マクロは引数を2つ取ります.
第1引数で与えられたマクロに第2引数で与えられた値を代入します.
第2引数はptやmmなどの大きさや長さについての単位を持った数値でなければなりません.
\verb|\addtolength|マクロは\verb|\setlength|マクロと同様に引数を2つ取ります.
第1引数で与えられたマクロに第2引数で与えられた値を加算します.
第2引数の制約は\verb|\setlength|と同様です.

\reflisting{list:layout}の1行目と2行目が``本文''の大きさの設定にあたります.
\begin{verbatim}
\setlength{\textwidth}{155truemm}
\setlength{\textheight}{230truemm}
\end{verbatim}
``本文''の幅を155mmに,高さを230mmに設定していることが分かります.
それぞれの単位に対して\texttt{true}がついているところが一つのミソです.
\texttt{jsbook}を用いてタイプセットを行うときは,はじめに欧文書体を10ptとして出力したあとに
オプションで指定されたフォントサイズに合わせて拡大縮小を行います.
\texttt{true}をつけることでこの拡大縮小の対象にならず,希望通りの長さを指定することができます.

``本文''の大きさが決めると,次はどこに配置するのかを決めなければなりません.
\texttt{jsbook}の標準設定では``本文''は印刷するページの左側から
1\ inch + \verb|\hoffset|というマージンを取って配置されます.
このマージンを調整するために使用するのが\verb|\oddsidemargin|と\verb|\evensidemargin|です.
奇数ページでは\verb|\oddsidemargin|が,偶数ページでは\verb|evensidemargin|がそれぞれ先に示した式に加算された値が最終的なページ左側から``本文''までのマージンになります.
ここで少し問題となるのが,ページの左側からのマージンだということです.
日本語書籍のレイアウトでは天,地,ノド,小口のそれぞれのアキについて指定します.
天はページ上部の,地はページ下部の,ノドはページの綴じ側の,小口はページの開き側のことを指します.
小口アキが狭い時はページを開く際に,開く指に本文が隠れてしまいます.
ノドアキが狭い時はページを開いた時に綴じ側に隠れて本文が読みづらくなってしまいます.
そこでこれらを適切に設定することは本の読みやすさを左右する上で重要です.
ところが{\pLaTeX}を用いる場合は用紙からの左マージンしか指定できないため,
偶数ページでは小口アキが,奇数ページではノドアキが指定されることになります.
\reflisting{list:layout}の4行目から7行目が``本文''の左マージンの調整を行っています.

\begin{verbatim}
\setlength{\oddsidemargin}{15truemm}
\addtolength{\oddsidemargin}{-1.15truein}
\setlength{\evensidemargin}{15truemm}
\addtolength{\evensidemargin}{-1.15truein}
\end{verbatim}

この設定では奇数ページと偶数ページの左マージンをそれぞれ同一にしています.
単位が混在しているためmmに変換して計算すると,左マージンはおおよそ-14.21mmとなります.
計算してみると奇数ページと偶数ページのノドと小口のアキは表\ref{tbl:oddevenmargins}のようになります.
\begin{table}[!ht]
	\centering
	\caption{偶数/奇数ページのノド/小口}
	\label{tbl:oddevenmargins}
	\begin{tabular}{c|cc} \hline \hline
		& ノド & 小口 \\ \hline
		偶数ページ & 15.81mm & 11.19mm \\
		奇数ページ & 11.19mm & 15.81mm \\ \hline
	\end{tabular}
\end{table} 

というわけで,とよぎぃ通信では奇数ページと偶数ページでノドと小口が揃っていないということがよく分かります.
% 今までのとよぎぃ通信は,奇数ページの方が綴じ側のアキが狭くなっているはずですが,気づかれた方はいるのでしょうか.
% 本当に?要検証
% 多分だけどmarginnotesの部分がどうなるかが設定によって変わりそう?
% jsbookだと10ptと11ptではmarginnotesのとり方が違う.11ptだとmarginnotesのアキが用意されない
% ということは実質的にmarginnotesがないも同じことになる.
% 10ptだと左右で異なるページ組,11ptだと左右ページとも変わらないページ組(天地左右中央)

同様に{\pLaTeX}では1 inch + \verb|\voffset|だけページ上部だけマージンを取って``ヘッダー''が配置されます.
この``ヘッダー''とページ上部とのマージンを調整するために\verb|\topmargin|が用いられます.
そして``本文''は``ヘッダー''から\verb|\headsep|だけマージンを取って配置されます.
また``ヘッダー''の高さは\verb|\headheight|て定義されているため,
ページ上部から``本文''までのアキは1 inch + \verb|\voffset| + \verb|\headsep| + \verb|\headheight| + \verb|\topmargin|ということになります.
\reflisting{list:layout}の8,9行目で\verb|\topmargin|について設定を行っています.
\begin{verbatim}
\setlength{\topmargin}{30truemm}
\addtolength{\topmargin}{-2.4truein}
\end{verbatim}
\verb|\topmargin|を単位系を直して計算してみるとおおよそ-30.96mmとなります.
これを先ほどの式に当てはめます. 
\verb|jsbook|では\,\verb|\voffset|が0pt, \verb|\headsep|が17pt, \verb|\headheight|が20pt
となっているのでこれらの単位系を揃えて計算するとおおよそ7.45mmとなります.
標準の設定ですとおおよそ36.3mm確保されていることになるため,かなりページ上部のアキがつめられています.

また,ここから``本文''とページ下端の間に存在するアキも導出できます.
ページサイズがB5(JIS)のため,縦の長さは257mmです.
``本文''の縦の長さが230mm,ページ上端と``本文''のアキがおおよそ7.45mmのため,$257 - 230 - 7.45 \simeq 19.55\mathrm{mm}$が``本文''とページ下端の間に存在するアキと求まります.

ページレイアウトについて得られたことをまとめてみましょう.
ページレイアウトの設定では``本文''の大きさとそれに伴って変更された``ヘッダー'',\,``フッター''のマージンについて見てきました.
結果としてとよぎぃ通信第3号は次のようにページレイアウトが設計されていたことが明らかとなりました.
表\ref{tbl:toyogylayout}にそれぞれのアキをまとめて示します.

\begin{table}[!ht]
	\centering
	\caption{とよぎぃ通信の版面設計}
	\label{tbl:toyogylayout}
	\begin{tabular}{c|cc|c} \hline \hline
		& 偶数ページ & 奇数ページ & \verb|jsbook|のレイアウト(11pt) \\
		天 & 7.45mm & 7.45mm & 33.5mm \\
		地 & 19.55mm & 19.55mm & 40.0mm \\
		ノド & 15.81mm & 11.19mm & 20.5mm \\
		小口 & 11.19mm & 15.81mm & 31.8mm \\ \hline
	\end{tabular}
\end{table}

こうして見れば,\,\verb|jsbook|の標準に比べて天地ノド小口すべてのアキが狭くなっていることが見て取れます.
とよぎぃ通信では書きたいことを前面に押し出すために,\,``本文''を広く取っていると考えられます.
ただし真相は杳として知れません.

さて, \reflisting{list:layout}の設定のなかでひとつだけ解説していないものがあります.
それが3行目のマクロです.\,
\verb|\fullwidth|は\verb|jsclasses|で定義されているマクロです.
一般的に書籍の組版を行う場合,本文の1行は全角40文字と設定されることが多いです.
この場合,全角40文字に収まるように左右のマージンを調整するわけですが,``ヘッダー''や``フッター''は本文領域の幅よりも広く取りたいということが考えられます.
そのために用意されているのがこの\verb|\fullwidth|です.\,
\verb|jsbook|では紙幅から36mmを引いた値が\verb|\fullwidth|に設定されることになっていますが,とよぎぃ通信では``本文''の幅と同じ値としています.
つまり``ヘッダー''と``フッター''の幅が``本文''と同じになっています.
ただ,それを実感することはなかなか難しいと思います.

さらに``フッター''にも少しだけ手が加えられています.
\begin{lstlisting}[caption = ``フッター''の設定, label = list:footer, language=tex]
\makeatletter
\def\ps@plainfoot{%
\let\@mkboth\@gobbletwo
\let\@oddhead\@empty
\def\@oddfoot{\normalfont\hfil-- \thepage\ --\hfil}%
\let\@evenhead\@empty
\let\@evenfoot\@oddfoot}
\let\ps@plain\ps@plainfoot
\makeatother
\end{lstlisting}
``フッター''の設定で意味をなしているのは\texttt{\def\@oddfoot{\normalfont\hfil-- \thepage\ --\hfil}}のみになります.
というのも,それ以外の設定は\texttt{jsbook}によって指定されているものと寸分違わぬものとなっているためです.
抜き出した設定で行っているのはページ番号をフッターの中央揃えにして,ページ番号の両端にエンダッシュ(--)を追加する設定となっています.

\subsection{図や表の挿入についての設定}

{\pLaTeX}では図や表はフロートと呼ばれる領域を``本文''内部に作成した上で,フロート内に配置されます.
\reflisting{list:float}ではフロートが挿入された時の見た目に関する設定を行っています..
\begin{lstlisting}[caption = フロートついての設定, label = list:float, language = tex]
\setlength{\intextsep}{10.5pt}
\setlength{\textfloatsep}{10.5pt}

\renewcommand{\figurename}{Fig.}
\renewcommand{\tablename}{Table }

\makeatletter
\setlength{\abovecaptionskip}{0pt}
\long\def\@makecaption#1#2{%
\vskip\abovecaptionskip
\iftdir\sbox\@tempboxa{#1\hskip1zw#2}%
\else\sbox\@tempboxa{#1 #2}%
\fi
\ifdim \wd\@tempboxa >\hsize
\iftdir #1\hskip1zw#2\relax\par
\else #1 #2\relax\par\fi
\else
\global \@minipagefalse
\hbox to\hsize{\hfil\box\@tempboxa\hfil}%
\fi
\vskip\belowcaptionskip}
\makeatother
\end{lstlisting}

1行目から順番に見ていきます.
\verb|\textfloatsep|は本文領域の下端にフロートを挿入した時に,
本文とフロートの間に挿入されるアキを指定しています.
この場合はフロートの上部にのみアキが挿入されます.
\verb|\intextsep|は本文領域内にフロートを挿入した時に,
本文とフロートの間に挿入されるアキについて指定を行います.
図や表を挿入するときのオプションに\texttt{[h]}を指定した時に
挿入されるアキであると考えてもいいでしょう.
この場合はフロートの前後に指定したアキが挿入されます.

この設定では\texttt{jsbook}の標準設定に比べてアキを狭くしています.
特に,ページ下端にフロートが挿入される場合のアキは標準設定に比べて
半分ほどのアキになっています.

次に\verb|makeatletter|$\sim$\verb|makeatother|で囲まれた範囲の設定について見ていきます.
重要な部分は\verb|\long\def\@makecaption|によって定義される\verb|@makecaption|マクロです.
マクロ定義のプレフィックスとして指定されている\verb|\long|はマクロによって展開される文字列内に改行や段落が含まれることを意味しています.
少しわかりづらい部分があるため,一つ一つほぐしながら見ていきます.
まず,はじめに\verb|@makecaption|は引数を2つ取ります.
\verb|#1|はキャプションの番号が,\verb|#2|にはキャプションとして表示するテキストが入ります.

\begin{verbatim}
\vskip\abovecaptionskip
\end{verbatim}%
は{\TeX}のプリミティブマクロ\verb|\vskip|を呼び出しています\footnote{正確にはprimitive control sequence}\footnote{\url{https://www.tug.org/utilities/plain/cseq.html#vskip-rp}}.
これは引数を一つ取るプリミティブマクロで,垂直方向のアキを挿入します.
ところで\verb|\abovecaptionskip|は\verb|@makecaption|の定義の直前に\verb|\setlength|によってその大きさを0ptにされています.
このことからフロートに表題を加える際に表題の上部にはアキを追加しないことを意味します.

\begin{verbatim}
\iftdir\sbox\@tempboxa{#1\hskip1zw#2}%
\else\sbox\@tempboxa{#1 #2}%
\fi
\end{verbatim}%
は{\TeX}マクロに慣れていないとややこしく見えます.
\verb|\iftdir|は{p\TeX}の拡張プリミティブマクロで,現在の文字組の方向が縦向きである時に真を取ります\footnote{\url{https://osdn.net/projects/eptex/docs/tc16ptex/ja/1/tc16ptex.pdf}}
\footnote{\url{http://d.hatena.ne.jp/zrbabbler/20160612/1465700860}}.
とよぎぃ通信では全ページ横組となっているため,なぜこのマクロが加えられているのかは不明ですが利用されています.
\verb|\iftdir|によらず\verb|\sbox|を呼び出しています.
これは{\LaTeX}拡張のマクロの一つであり,\verb|\sbox{cmd}{text}|という形式になっています.
このマクロでは,\verb|{text}|が入る大きさのボックスを作成し,
その中にテキストを実際に流しこむ代わりに\verb|{cmd}|で指定されたマクロにボックスの情報を保存します.
ここでは\verb|@tempboxa|というマクロにキャプションを作るときに指定されたテキストが収まるようなボックスの情報が保存されます.\\
また,縦組みの時にボックスを作成するにあたって\verb|\hskip1zw|という一見不可思議なプリミティブマクロが挿入されています.
\verb|\hskip|はvertical modeで用いられるとき,水平方向へのアキを挿入するのではなく
新たな段落を生成するという動作が定義されており,これを利用したものと考えられます.
作成された\verb|@tempboxa|を使用するのが次の\verb|\ifdim|から始まるマクロです.
\footnote{というのも{p\TeX}内部において\texttt{\textbackslash{}tate}がどのように定義されているかを確認していないためである}

この部分の解説は次の2つを詳しく読んだほうが良い
% ftp://ftp.ccu.edu.tw/pub/tex/language/japanese/ptex-base/ptexdoc.tex
% file:///home/um-akahana/Downloads/doronawa0.4.pdf
\begin{verbatim}
\ifdim \wd\@tempboxa >\hsize
\iftdir #1\hskip1zw#2\relax\par
\else #1 #2\relax\par\fi
\else
\global \@minipagefalse
\hbox to\hsize{\hfil\box\@tempboxa\hfil}%
\fi
\end{verbatim}
\verb|\ifdim|は{\TeX}のプリミティブマクロで次元を持つ2つの値を指定された比較演算子で比較する.
ここで言う次元はmmやptなどの組版を行う際に用いる大きさや長さを示す単位を指します.
\verb|\ifdim|ではキャプションが入るボックスの長さが組方向の行の長さを超えていないか確認している.
\verb|\wd|は{\TeX}のプリミティブマクロで引数として与えられたボックスの横幅を取得します.
% hsizeは水平方向の長さの可能性がある.\TeXの標準は左から右に字が送られていって行を形成し
% 行は上から下に送られていくことでページを作成する
\verb|\hsize|は行の長さを示しているため,キャプションが行の長さより大きくなっていないかという判定をしています.
行の長さよりもキャプションが入るボックスが大きくなっている場合は,キャプションをなるべく表示するためにキャプションをページに対して中揃えしないようにしています.
\verb|\relax|は何もしないということを意味する{\TeX}のプリミティブマクロです.
また,\verb|\par|は改段落を行う{\TeX}のプリミティブマクロです.

キャプションが行の長さを超えない場合はそのまま出力すれば良いということになります.
この時は,キャプションの長さが組方向の行の長さを超えないために,キャプションをセンタリングして出力しています.
\verb|\hbox|は{\TeX}のプリミティブマクロで水平方向のボックスを作成します.
ここでは引数として\verb|to\hsize|を与えているため行の長さ

\subsection{数式についての設定}
\begin{lstlisting}[caption = 数式まわりの設定, label = list:eq, language = tex]
\makeatletter
\@addtoreset{equation}{section}
\def\theequation{\thesection.\arabic{equation}}
\makeatother
\end{lstlisting}

数式についての設定はフロートについての設定に比べるとかなり簡単なものです.
l行目と5行目は,フロートについての設定でも登場したある種の定型句です.

\section*{参考文献}
\url{http://www.biwako.shiga-u.ac.jp/sensei/kumazawa/tex.html}
